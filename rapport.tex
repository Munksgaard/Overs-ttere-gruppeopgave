\documentclass [10pt,a4paper]{article}
\usepackage[danish]{babel}
\usepackage{a4wide}
\usepackage[T1]{fontenc}
\usepackage[utf8x]{inputenc}
\usepackage{amsmath}
\usepackage{amsfonts}
\usepackage{fancyhdr}
\usepackage{ucs}
\usepackage{graphicx}
\usepackage{listings}
\usepackage{color}

\pagestyle{fancy}
\fancyhead[LO]{Michael Thuling \& Philip Munksgaard}
\fancyhead[RO]{Oversættere: G-Opgave}
\fancyfoot[CO]{\thepage}


\title{G-Opgave}
\author{Michael Thulin \& Philip Munksgaard}

\begin{document}
\maketitle

\section{Tilføjelser til Lexer.lex}

Vi har tilføjet de manglende keywords og tokens til Lexer.lex. 

\section{Parser.grm}

Her startede vi med at tilføje de manglende tokens i
grammatikken. Derefter tilføjede vi de manglende typer og så de
manglende produktioner. Så endte vi op med mere end 15 shift/reduce
konflikter. For at slippe af med nogle af disse, eliminerede vi
venstre-rekursion i adskillige af produktionerne, heriblandt Types,
Pats og Exps. I nogle produktionerne, f.eks. Dec og Match, valgte vi
at elmiminere tvetydighed ved at sætte associativitet korrekt.
Den eneste shift/reduce-fejl der voldte os nævneværdige problemer var
tvetydighed omkring $LPAR Exp RPAR$, som vi valgte at løse ved at
dele produktionen $Exp \to LPAR Exps RPAR COLON ID$ op i to
produktioner, $Exp \to LPAR Exp RPAR COLON ID$ og $Exp \to LPAR Exp
COMMA Exps RPAR COLON ID$, jvf. bilag 1.

Parser.grm oversætter nu uden fejl.

\section{Type.sml}

Vi har lavet de nødvendige tilføjelser i checkType, checkPat,
checkExp. Derudover tilføjede vi kode til at hente type-deklarationer
ud af tyDecs. 

Herunder ses pseudokode for checkExp, checkDec og checkTuple, som er
de dele af Type.sml vi har lavet mest om i.

\begin{tabular}{|l|l|}
  \hline
  \multicolumn{2}{|l|}{checkExp(Exp, vtable, ftable, ttable) = case
    Exp of} \\ \hline
   true & bool \\ \hline
  false & bool \\ \hline
  null & t = lookup x ttable \\
  & if t = unbound \\
  & then error() \\
  & else TyVar s \\ \hline
  id & t = lookup x vtable \\
  & if t = unbound \\
  & then error() \\
  & else t \\ \hline
  Exp1 = Exp2 & t1 = checkExp(Exp1, vtable, ftable, ttable) \\
  & t2 = checkExp(Exp1, vtable, ftable, ttable) \\ 
  & if t1 = int and t2 = int \\
  & then bool \\
  & else error() \\ \hline
  if Exp1 & t1 = checkExp(Exp1, vtable, ftable, ttable) \\
  then Exp2 & t2 = checkExp(Exp2, vtable, ftable, ttable) \\
  else Exp3 & t3 = checkExp(Exp3, vtable, ftable, ttable) \\
  & if t1 = bool and t2 = t3 \\
  & then t2 \\
  & else error() \\ \hline
  Exp1 < Exp2 & t1 = checkExp(Exp1, vtable, ftable, ttable) \\
  & t2 = checkExp(Exp1, vtable, ftable, ttable) \\ 
  & if t1 = int and t2 = int \\
  & then bool \\
  & else error() \\ \hline
  not Exp1 & t1 = checkExp(Exp1, vtable, ftable, ttable) \\
  & if t1 = bool \\
  & then bool \\
  & else error() \\ \hline
  Exp1 and Exp2 & t1 = checkExp(Exp1, vtable, ftable, ttable) \\
  & t2 = checkExp(Exp1, vtable, ftable, ttable) \\ 
  & if t1 = bool and t2 = bool \\
  & then bool \\
  & else error() \\ \hline
  Exp1 or Exp2 & t1 = checkExp(Exp1, vtable, ftable, ttable) \\
  & t2 = checkExp(Exp1, vtable, ftable, ttable) \\ 
  & if t1 = bool and t2 = bool \\
  & then bool \\
  & else error() \\ \hline
  let id = Dec1 & t1 = checkDec(Dec1, vtable, ftable, ttable) \\
  in Exp1 & checkExp(Exp1, vtable, ftable, ttable) \\ \hline
  (Exps) : id & ty = lookup s ttable \\
  & if ty = unbound \\
  & then error() \\
  & else checkTuple Exps ty vtable ftable ttable \\
  & TyVar id \\ \hline
  case Exp1 of & ty = checkExp(Exp1, vtable, ftable, ttable) \\
  in Match1 end & checkMath m ty vtable ftable ttablee pos \\ \hline
\end{tabular}

\begin{tabular}{|l|l|}
  \hline
  \multicolumn{2}{|l|}{checkDec (decs, vtable, ftable, ttable) = case
    decs of} \\ \hline
  (pat, e) & ty = checkExp e vtable ftable ttable \\
  & vtable1 = checkPat p ty ttable \\
  vtable1 @ vtable & \\ \hline
  (pat, e) :: decs & ty = checkExp e vtable ftable ttable \\
  & vtable1 = checkPat pat ty ttable \\
  & td = checkDec (decs, (vtable1 @ vtable), ftable, ttable) \\
  & vtable1 @ td @ vtable \\ \hline
\end{tabular}

\begin{tabular}{|l|l|}
  \hline
  \multicolumn{2}{|l|}{checkTuple(e, ty, vtable, ftable, ttable) = case (e, ty) of} \\ \hline
  	(e, ty) & If checkExp e vtable ftable ttable = ty \\
            & then [ty] \\
        	& else error() \\ 
			\hline
	(e :: es, ty :: tys) & if checkExp e vtable ftable ttable  = ty \\
			        	 & then ty :: checkTuple es tys vtable ftable ttable \\
			             & else error() \\ \hline
\end{tabular}

\section{Compiler.sml}

Nedenfor kan ses pseudokode for hvordan koden for de nye
konstruktioner genereres.

\begin{tabular}{|l|l|}
  \hline
  \multicolumn{2}{|l|}{CompilePat (p, v, vtable, fail) = case p of} \\
	\hline
	TrueP & [place := 1] \\
	\hline
	FalseP & [place := 0] \\ 
	\hline
	NullP & [place := 0] \\ 
	\hline
	TupleP & ($code_1$, $vtable_1$) = compilePats(pats, v, vtable, fail, 0) \\ 
		   & if (v = 0) \\
		   & then Error() \\
		   & else (code1, vtable) \\
	\hline
\end{tabular}
\\
\begin{tabular}{|l|l|}
  \hline
  \multicolumn{2}{|l|}{CompilePats (pats, v, vtable, fail, offset) = case pats of} \\
	\hline
	([], []) & ([], []) \\
	\hline
	(pat :: pats) & x = newvar() \\
	& $code_1$ = LW(x, v, offset) \\
	& $(code_2, vtable_1)$ = compilePat(pat, x, vtable, fail) \\
	& $(code_3, vtable_2)$ = compilePats(pats, v, ($vtable_1$ @ $vtable_2$), fail, (offset + 4)) \\
	& ($code_1$ @ $code_2$ @ $code_3$, $vtable_2$ @ $vtable_1$ @ $vtable$) \\
	\hline
\end{tabular}

\begin{tabular}{|l|l|}
  \hline
  \multicolumn{2}{|l|}{compileDec(dec, vtable, fail) = case dec of}
  \\ \hline
  [] & ([], []) \\ \hline
  (p, e) :: ds & t := newvar() \\
  & code1 := compileExp e vtable t \\
  & (code2, vtable1) := compilePat p t vtable fail \\
  & (code3, vtable2) := compileDec ds (vtable1 @ vtaale) fail \\
  & (code1 @ code2 @ code2, vtable2 @ vtable 1 vtable) \\ \hline
\end{tabular}

\begin{tabular}{|l|l|}
  \hline
  \multicolumn{2}{|l|}{compileDec(exps, vtable, fail) = case exps of}
  \\ \hline
  [] & [] \\ \hline
  & t1 = newvar() \\
  & t2 = newvar() \\
  & code1 = compileExp e vtable t1 \\
  & code2 = compileTuple ees vtable mem \\
  & code1 @ [SW (t1, mem, 0), mem := mem + 4] @ \\
  & code2 \\ \hline
\end{tabular}

\section{Fejl og mangler}

Vores compiler kompilerer alle test-filerne korrekt. Der er dog et par
af error-filerne som ikke genererer fejl. Det gælder error14.cat,
error17.cat, error18.cat og error19.cat. Fejlende skyldes manglende
tjek i Type.sml.

\newpage 

\section{Bilag 1: Produktioner for Exp}

\begin{lstlisting}[frame=single,language=Clean]    
Exp :	  NUM		{ Cat.Num $1 }
        | TRUE          { Cat.True $1 }
        | FALSE         { Cat.False $1 }
        | NULL COLON ID { Cat.Null (#1 $3, $1) }
	| ID            { Cat.Var $1 }
        | LPAR Exp COMMA Exps RPAR COLON ID
                        { Cat.MkTuple ( $2 :: $4, #1 $7, $1 ) }
        | LPAR Exp RPAR COLON ID 
                        { Cat.MkTuple ([$2], #1 $5, $1) }
	| Exp PLUS Exp	{ Cat.Plus ($1, $3, $2) }
	| Exp MINUS Exp	{ Cat.Minus ($1, $3, $2) }
        | Exp EQUAL Exp { Cat.Equal ($1, $3, $2) }
        | Exp LESSTHAN Exp
                        { Cat.Less ($1, $3, $2) }
        | NOT Exp       { Cat.Not ($2, $1) }
        | Exp AND Exp   { Cat.And ($1, $3, $2) }
        | Exp OR Exp    { Cat.Or ($1, $3, $2) }
        | IF Exp THEN Exp ELSE Exp
                        { Cat.If ($2, $4, $6, $1) }
        | LET Dec IN Exp
                        { Cat.Let ($2, $4, $1) }
        | CASE Exp OF Match END
                        { Cat.Case ($2, $4, $1) }
	| ID Exp %prec WRITE
			{ Cat.Apply (#1 $1, $2, #2 $1) }
	| READ		{ Cat.Read $1 }
	| WRITE Exp	{ Cat.Write ($2, $1) }
	| LPAR Exp RPAR { $2 }
;
\end{lstlisting}

\end{document}
